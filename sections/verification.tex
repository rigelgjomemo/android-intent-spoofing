% !TEX root =  ../main.tex
\section{Verification and Exploit Generation}
\label{sec:verification}

Once a path is marked as vulnerable, we are interested in demonstrating its actual 
exploitability. The fact that a path simply exists does not imply that there exists a 
tuple of values that can actually lead to the final vulnerable sink. For this reason, 
we complemented our tool with an exploit generator, able to generate an example 
command that triggers the vulnerability.

\subsection{String solving}


We then parse the solutions obtained, if any, by querying the Kaluza solver, and we relate them to the corresponding IFDS fact.

\subsection{Path validation}

\subsection{Exploit String Generation}
