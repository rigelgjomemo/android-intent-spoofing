% !TEX root =  ../main.tex
\subsection{Attacker construction}
In this section we highlight the implementation-related details of an attacking application.
An effective attack can be automatically prepared in the form of a well-crafted application able to send
Intent messages to the right target, at the right moment in time and carrying the malicious payload.
We formulate the attack application as a simple utility application (such as a torch), appealing to our victims,
 that includes a malicious \emph{service}.

Our malicious service embeds the exploit strings obtained from the previous steps in a list of pre-populated
Intent messages ready to be sent, one for each demonstrated vulnerability.
In order for the attack to be successful we want to enhance the service logic and domain knowledge to
obtain the best attack scenario possible: the user perception of the environment during the attack must
be as transparent as possible.

For each of the messages, the service first needs to check whether the target vulnerable application
is installed on the system or not. This is possible through the Android \emph{Package Manager API}\cite{android-package-manager} offering
the $getInstalledApplications$ method. The invocation with the $PackageManager.GET\_META\_DATA$ specified as argument
returns a complete list of all the applications installed on the phone. The list contains several meta-data such as
the application's package name or the application's launch Intent.

Once it is ensured that the application we want to target is present on the system, we can do one step
further by checking if the application is currently active (in foreground) on the user phone.

This is particularly important for two reasons: first we limit the context switching consequences of presenting to the user a completed unrelated content (e.g. a different application screen), then we can assume
that the user uses the vulnerable application and hence that the user is logged in at the moment of the attack.
Also in this case, Androids offers an API: the \emph{Activity Manager API}\cite{android-activity-manager}, in particular the $getRunningAppProcesses$ method that returns a list of $RunningAppProcessInfo$ containing the foreground/background information along with the packages Application process package information, that we
use to identify the application.

When ensured that a victim application is installed and running, the attacker can easily send the malicious Intent message to trigger the attack.

By separating this two check we can leave our application silent, avoiding to poll the running processes,
if none of the vulnerable applications is currently installed on the device. Indeed, since application
installations are not so common, the installation check can be performed in reasonably long time intervals.
